\documentclass[spec, och, labwork]{shiza}
% параметр - тип обучения - одно из значений:
%    spec     - специальность
%    bachelor - бакалавриат (по умолчанию)
%    master   - магистратура
% параметр - форма обучения - одно из значений:
%    och   - очное (по умолчанию)
%    zaoch - заочное
% параметр - тип работы - одно из значений:
%    referat    - реферат
%    coursework - курсовая работа (по умолчанию)
%    diploma    - дипломная работа
%    pract      - отчет по практике
% параметр - включение шрифта
%    times    - включение шрифта Times New Roman (если установлен)
%               по умолчанию выключен
\usepackage{subfigure}
\usepackage{tikz,pgfplots}
\pgfplotsset{compat=1.5}
\usepackage{float}

%\usepackage{titlesec}
\setcounter{secnumdepth}{4}
%\titleformat{\paragraph}
%{\normalfont\normalsize}{\theparagraph}{1em}{}
%\titlespacing*{\paragraph}
%{35.5pt}{3.25ex plus 1ex minus .2ex}{1.5ex plus .2ex}

\titleformat{\paragraph}[block]
{\hspace{1.25cm}\normalfont}
{\theparagraph}{1ex}{}
\titlespacing{\paragraph}
{0cm}{2ex plus 1ex minus .2ex}{.4ex plus.2ex}

% --------------------------------------------------------------------------%


\usepackage[T2A]{fontenc}
\usepackage[utf8]{inputenc}
\usepackage{graphicx}
\graphicspath{ {./images/} }
\usepackage{tempora}

\usepackage[sort,compress]{cite}
\usepackage{amsmath}
\usepackage{amssymb}
\usepackage{amsthm}
\usepackage{fancyvrb}
\usepackage{listings}
\usepackage{listingsutf8}
\usepackage{longtable}
\usepackage{array}
\usepackage[english,russian]{babel}

% \usepackage[colorlinks=true]{hyperref}
\usepackage{url}

\usepackage{underscore}
\usepackage{setspace}
\usepackage{indentfirst} 
\usepackage{mathtools}
\usepackage{amsfonts}
\usepackage{enumitem}
\usepackage{tikz}
\usepackage{minted}

\newcommand{\eqdef}{\stackrel {\rm def}{=}}
\newcommand{\specialcell}[2][c]{%
\begin{tabular}[#1]{@{}c@{}}#2\end{tabular}}

\renewcommand\theFancyVerbLine{\small\arabic{FancyVerbLine}}

\newtheorem{lem}{Лемма}

\begin{document}

% Кафедра (в родительном падеже)
\chair{}

% Тема работы
\title{Теория псевдослучайных генераторов}

% Курс
\course{4}

% Группа
\group{431}

% Факультет (в родительном падеже) (по умолчанию "факультета КНиИТ")
\department{факультета КНиИТ}

% Специальность/направление код - наименование
%\napravlenie{09.03.04 "--- Программная инженерия}
%\napravlenie{010500 "--- Математическое обеспечение и администрирование информационных систем}
%\napravlenie{230100 "--- Информатика и вычислительная техника}
%\napravlenie{231000 "--- Программная инженерия}
\napravlenie{100501 "--- Компьютерная безопасность}

% Для студентки. Для работы студента следующая команда не нужна.
% \studenttitle{Студентки}

% Фамилия, имя, отчество в родительном падеже
\author{Ивановой Ксении}

% Заведующий кафедрой
% \chtitle{} % степень, звание
% \chname{}

%Научный руководитель (для реферата преподаватель проверяющий работу)
\satitle{доцент} %должность, степень, звание
\saname{И. И. Слеповичев}

% Руководитель практики от организации (только для практики,
% для остальных типов работ не используется)
% \patitle{к.ф.-м.н.}
% \paname{С.~В.~Миронов}

% Семестр (только для практики, для остальных
% типов работ не используется)
%\term{8}

% Наименование практики (только для практики, для остальных
% типов работ не используется)
%\practtype{преддипломная}

% Продолжительность практики (количество недель) (только для практики,
% для остальных типов работ не используется)
%\duration{4}

% Даты начала и окончания практики (только для практики, для остальных
% типов работ не используется)
%\practStart{30.04.2019}
%\practFinish{27.05.2019}

% Год выполнения отчета
\date{2024}

\maketitle

% Включение нумерации рисунков, формул и таблиц по разделам
% (по умолчанию - нумерация сквозная)
% (допускается оба вида нумерации)
% \secNumbering

%-------------------------------------------------------------------------------------------
\tableofcontents

\section{Генераторы псевдослучайных чисел}
\subsection{Линейный конгруэнтный метод}
Линейной конгруэнтной последовательностью (ЛКП) называется последовательность ПСЧ, получаемая по формуле:
\begin{equation}
  X_{n + 1} = (aX_n + c) \mod m, \; n \geq 1,
\end{equation}

Параметры запуска программы:
Модуль, множитель, приращение, начальное значение.

Пример запуска:
\begin{small}
\begin{verbatim}
  python3 main.py -g lc -i 1024,1664525,1013904223,1 -n 10000 -f lc.dat
\end{verbatim}
\end{small}

Алгоритм:
\begin{small}
\begin{verbatim}
def lc(args=args_for_gen["lc"]):
  a = int(args[0])
  c = int(args[1])
  mod = int(args[2])
  x_n = int(args[3]) 
  while(True):
      x_n = (x_n * a + c) % mod
      yield x_n
\end{verbatim}
\end{small}


% -------------% -------------% -------------% -------------% -------------% -------------

\subsection{Аддитивный метод}

Последовательность определяется следующим образом:
\begin{equation}
  X_n = (X_{n - k} + X_{n - j}) \mod m, \; j > k \geq 1.
\end{equation}

Параметры запуска программы: Модуль, младший индекс, старший индекс, последовательность начальных
значений.

Пример запуска:
\begin{small}
\begin{verbatim}
  python3 main.py -g add -i 30000,24,55,79,134,213,347,560,907,1467,2374,3...,
  17506 -n 10000 -f add.dat
\end{verbatim}
\end{small}

Алгоритм:
\begin{small}
\begin{verbatim}
def add(args=args_for_gen["add"]):
  mod, k, j = int(args[0]), int(args[1]), int(args[2])
  x = list(map(int, args[3:]))
  i = 55
  while(True):
      val = (x[i-k] + x[i-j]) % mod
      x.append(val)
      i += 1
      yield val
\end{verbatim}
\end{small}


% -------------% -------------% -------------% -------------% -------------% -------------

\subsection{Пятипараметрический метод}

Метод является частным случаем РСЛОС, использует характеристический многочлен из 5 членов и позволяет генерировать последовательности $w$-битовых двоичных целых чисел в соответствии со следующей рекуррентной 
формулой:

\begin{equation}
  X_{n + p} = X_{n + q_1} + X_{n + q_2} + X_{n + q_3} + X_n, \; n = 1, 2, 3, \dots
\end{equation}

Параметры запуска программы: p, q1, q2, q3, w, начальное значение.

Пример запуска:
\begin{small}
\begin{verbatim}
  python3 main.py -g 5p -i 89,7,13,24,10,234122131 -n 10000 -f 5p.dat
\end{verbatim}
\end{small}

Алгоритм:
\begin{small}
\begin{verbatim}
def fp(args=args_for_gen["fp"]):
  p, w, x = int(args[0]), int(args[4]), int(args[5])
  mask_w, mask_p = 0, 0
  list_par = list(map(int, args[1:4]))
  list_par.append(0)

  for i in range(w):
      mask_w = set_bit(mask_w, i, 1)
  for i in range(p):
      mask_p = set_bit(mask_p, i, 1)

  while(True):
      cur_bit = 0
      for i in list_par:
          cur_bit ^= get_bit(x, i)
      x = shift(x, p)
      x = set_bit(x, 0, cur_bit)
      x = x & mask_p
      yield x & mask_w
\end{verbatim}
\end{small}



% -------------% -------------% -------------% -------------% -------------% -------------

\subsection{Регистр сдвига с обратной связью (РСЛОС)}

Регистр сдвига с обратной линейной связью (РСЛОС) "--- регистр сдвига битовых слов, у которого входной (вдвигаемый) бит является линейной функцией остальных битов. Вдвигаемый вычисленный бит заносится в 
ячейку с номером 0. Количество ячеек $p$ называют длиной регистра.

Одна итерация алгоритма, генерирующего последовательность, состоит 
из следующих шагов:
\begin{enumerate}
  \item Содержимое ячейки $p - 1$ формирует очередной бит ПСП битов.
  \item Содержимое ячейки 0 определяется значением функции обратной связи, 
  являющейся линейной булевой функцией с коэффициентами $a_1, \dots, a_{p - 1}$.
  \item Содержимое каждого $i$-го бита перемещается в $(i + 1)$-й, $0 \leq i < p - 1$. 
  \item В ячейку 0 записывается новое содержимое, вычисленное на шаге 2.
\end{enumerate}

Параметры запуска программы: двоичное представление вектора коэффициентов, начальное значение регистра

Пример запуска:
\begin{small}
\begin{verbatim}
  python3 main.py -g lfsr -i 10011011010011010,000101000111 -n 10000 -f lfsr.dat
\end{verbatim}
\end{small}

Алгоритм:
\begin{small}
\begin{verbatim}
def lfsr(args=args_for_gen["lfsr"]):
  s, x, y = len(args[1]), int(args[0], 2), int(args[1], 2)
  while(True):
      current = 0
      for i in range(s):
          current ^= get_bit(x, i) * get_bit(y, i)
      y = shift(y, s)
      y = set_bit(y, 0, current)
      yield y
\end{verbatim}
\end{small}




% -------------% -------------% -------------% -------------% -------------% -------------

\subsection{Нелинейная комбинация РСЛОС}

Последовательность получается из нелинейной комбинации трёх РСЛОС следующим образом:
\begin{equation}
  f(x_1, x_2, x_3) = x_1 x_2 \oplus x_2 x_3 \oplus x_3
\end{equation}


Параметры запуска программы: двоичное представление векторов коэффициентов
для R1, R2, R3, w, x1, x2, x3. w – длина слова, x1, x2, x3 – десятичное представление
начальных состояний регистров R1, R2, R3.

Пример запуска:
\begin{small}
\begin{verbatim}
  python3 main.py -g nfsr -i 00000001010101,01011100000111101,010101001100000,
  0001001,10000010010,000000001,1001 -f nfsr.dat -n 10000
\end{verbatim}
\end{small}

Алгоритм:
\begin{small}
\begin{verbatim}
def nfsr(args=args_for_gen["nfsr"]):
  r1 = lfsr([args[0], args[3]])
  r2 = lfsr([args[1], args[4]])
  r3 = lfsr([args[2], args[5]])
  w1 = 0
  for i in range(int(args[6])):
      w1 = set_bit(w1, i, 1)
  while True:
      a, b, c = next(r1), next(r2), next(r3)
      yield (a ^ b + b ^ c + c) & w1
\end{verbatim}
\end{small}


% -------------% -------------% -------------% -------------% -------------% -------------

\subsection{Вихрь Мерсенна}

Метод основан на свойствах простых чисел Мерсенна и обладает рядом
достоинств относительно многих других ГПСЧ. «Вихрь» – это преобразование, которое обеспечивает
равномерное распределение ПСЧ.

Числом Мерсенна называется натуральное число, определяемое формулой
\begin{equation}
  M_n = 2^n - 1.
\end{equation}

Алгоритм состоит из следующих шагов:
\begin{enumerate}
  \item Инициализируются значения $u, h, a$ по формуле:

  $u:=(1,0,\dots,0)$ - всего $w-r$ бит, 
  \newline $h:=(0,1,\dots,1)$ - всего $r$ бит,

  $a:=(a_{w-1}, a_{w-2}, \dots, a_0)$ - последняя строка матрицы $А$.
  \item $X_0, X_1, \dots, X_{p-1}$ заполняются начальными значениями.
  \item Вычисляется $Y:=(y_0, y_1, \dots, y_{w-1}):=(X^r_n | X^l_{n+1})$.
  \item Вычисляется новое значение $X_i$:
  
  $X_n := X_{(n+q) \text{mod} p} \oplus (Y >> 1) \oplus a$, если младший бит $y_0=1$;

  $X_n := X_{(n+q) \text{mod} p} \oplus (Y >> 1) \oplus 0$, если младший бит $y_0=0$;
  \item Вычисляется $X_iT$:
  \newline$Y:=X_n, $
  \newline$Y:=Y \oplus (Y >> u), $
  \newline$Y:=Y \oplus ((Y << s) \cdot b),$
  \newline$Y:=Y \oplus ((Y << t) \cdot c), $
  \newline$Z:=Y \oplus (Y >> l).$

  $Z$ подается на выход, как результат.
  \item $n:=(n+1) \text{mod } p$. Переход на шаг 3.
\end{enumerate}


Параметры запуска программы: модуль, начальное значение x.

Пример запуска:
\begin{small}
\begin{verbatim}
  python3 main.py -g mt -i 1000,1234 -n 10000 -f mt.dat
 \end{verbatim}
\end{small}

Алгоритм:
\begin{small}
\begin{verbatim}
def mt(args=args_for_gen["mt"]):
  p, w, r, q, a, u = 624, 32, 31, 397, 2567483615, 11
  s, t, l, b, c = 7, 15, 18, 2636928640, 4022730752
  mod = int(args[0])
  MT = [int(args[1])]
  lower_mask = (1 << r) - 1
  upper_mask = (~lower_mask * -1) & w1
  w1 = 0
  ind = p

  for i in range(w):
      w1 = set_bit(w1, i, 1)

  for i in range(1, p):
      MT.append((MT[i - 1] ^ (MT[i - 1] >> 30)) + i)

  while(True):
      if (ind >= p):
          for i in range(p):
              x = (MT[i] & upper_mask) + (MT[(i + 1) % p] & lower_mask)
              xA = x >> 1
              if (x & 1):
                  xA ^= a
              MT[i] = MT[(i + q) % p] ^ xA
          ind = 0
      y = MT[ind]
      ind += 1
      y ^= (y >> u)
      y ^= (y << s) & b
      y ^= (y << t) & c
      y ^= (y >> l)
      yield y % mod

\end{verbatim}
\end{small}



% -------------% -------------% -------------% -------------% -------------% -------------

\subsection{RC4}

Алгоритм состоит из следующих шагов:
\begin{enumerate}
  \item Инициализация $S_i$.
  \item $i = 0, j = 0$.
  \item Итерация алгоритма:
    \begin{enumerate}
      \item $ans = 0$;
      \item $i = (i + 1) \mod 256$;
      \item $j = (j + S_i) \mod 256$;
      \item $Swap(S_i, S_j)$;
      \item $t = (S_i + S_j) \mod 256$;
      \item $K = S_t$.
      \item $ans = ans << 1 + K \& 1$
      \item Если количество бит $< w$, перейти к шагу б.
    \end{enumerate}
\end{enumerate}

Параметры запуска программы: 256 начальных значений.

Пример запуска:
\begin{small}
\begin{verbatim}
  python3 main.py -g rc4 -i 213,968,838,64,355,214,212,...,659,180,10 -f rc.dat 
  -n 10000
 \end{verbatim}
\end{small}

Алгоритм:
\begin{small}
\begin{verbatim}
def rc4(args=args_for_gen["rc4"]):
  k = list(map(int, args[:-2]))
  w = int(args[-1])
  l = len(k)
  s = [i for i in range(l)]
  i, j = 0, 0
  for i in range(l):
      j = (j + s[i] + k[i]) % l
      s[i], s[j] = s[j], s[i]
  while(True):
      num = 0
      for t in range(w):
          i = (i + 1) % l
          j = (j + s[i]) % l
          s[i], s[j] = s[j], s[i]
          num = set_bit(num, t, s[(s[i] + s[j]) % l] & 1)
      yield num
\end{verbatim}
\end{small}


% -------------% -------------% -------------% -------------% -------------% -------------
\subsection{ГПСЧ на основе RSA}

Алгоритм состоит из следующих шагов:
\begin{enumerate}
  \item Инициализация чисел: $n = pq$, где $p$ и $q$ простые числа, числа $e$: $1 < e < f$, НОД($e, f$) = 1, $f = (p - 1)(q - 1)$ и числа $x_0$ из интервала $[1, n - 1]$.
  \item \texttt{For i = 1 to w do}
        \begin{enumerate}
          \item $x_i \leftarrow x_{i-1}^e \mod n$.
          \item $z_i \leftarrow $ последний значащий бит $x_i$
        \end{enumerate}
  \item Вернуть $z_1, \dots, z_w$.
\end{enumerate}

Параметры запуска программы: модуль n, число e, w, начальное значение x.

Пример запуска:
\begin{small}
\begin{verbatim}
  python3 main.py -g rsa -i 10967,571,77,10 -n 10000 -f rsa.dat
 \end{verbatim}
\end{small}

Алгоритм:
\begin{small}
\begin{verbatim}
def rsa(args=args_for_gen["rsa"]):
  n = int(args[0])
  e = int(args[1])
  x = int(args[2])
  w = int(args[3])
  while(True):
      z = 0
      for i in range(w):
          x = x ** e % n
          z = set_bit(z, w - i - 1, x & 1)
      yield z
\end{verbatim}
\end{small}

% -------------% -------------% -------------% -------------% -------------% -------------

\subsection{Алгоритм Блюм-Блюма-Шуба}

Алгоритм состоит из следующих шагов:
\begin{enumerate}
  \item Инициализация числа: $n = 127 \cdot 131 = 16637$.
  \item Вычислим $x_0 = x^2 \mod n$, которое будет начальным вектором.
  \item \texttt{For i = 1 to w do}
        \begin{enumerate}
          \item $x_i \leftarrow x_{i-1}^2 \mod n$.
          \item $z_i \leftarrow $ последний значащий бит $x_i$
        \end{enumerate}
  \item Вернуть $z_1, \dots, z_w$.
\end{enumerate}

Параметры запуска программы: начальное значение x (взаимно простое с n).

Пример запуска:
\begin{small}
\begin{verbatim}
  python3 main.py -g bbs -i 15621,10 -n 10000 -f bbs.dat
 \end{verbatim}
\end{small}

Алгоритм:
\begin{small}
\begin{verbatim}
def bbs(args=args_for_gen["bbs"]):
  p, q = 127, 131
  n = p * q
  x, w = int(args[0]), int(args[1])
  while(True):
      z = 0
      for i in range(w):
          x = x * x % n
          z = set_bit(z, w - i - 1, x & 1)
      yield z
\end{verbatim}
\end{small}





\section{Преобразование ПСЧ к заданному распределению}
\subsection{Стандартное равномерное с заданным интервалом}

Формула: \[ Y = bU + a \]

Пример запуска:
\begin{small}
\begin{verbatim}
  python3 main1.py -d st -f 5p.dat -p1 1024 -p2 1664525 -p3 876 
 \end{verbatim}
\end{small}

Алгоритм:
\begin{small}
\begin{verbatim}
  def st(x, a, b, m):
    return a + U(x, m) * b
\end{verbatim}
\end{small}


\subsection{Треугольное распределение}

Формула:
\[ Y = a + b(U_1 + U_2 - 1) \]

Пример запуска:
\begin{small}
\begin{verbatim} 
  python3 main1.py -d tr -f 5p.dat -p1 1024 -p2 1664525 -p3 876  

 \end{verbatim}
\end{small}

Алгоритм:
\begin{small}
\begin{verbatim}
def tr(x, a, b, m):
  y = []
  for i in range(0, len(x)-1, 2):
      y.append(a + b * (U(x[i], m) + U(x[i+1], m) - 1))
  return np.array(y)
\end{verbatim}
\end{small}


\subsection{Общее экспоненциальное распределение}

Формула:
\[ Y = -b \ln(U) + a \]

Пример запуска:
\begin{small}
\begin{verbatim}
  python3 main1.py -d ex -f 5p.dat -p1 1024 -p2 1664525 -p3 876  
 \end{verbatim}
\end{small}

Алгоритм:
\begin{small}
\begin{verbatim}
  def ex(x, a, b, m):
    return a - b * np.log(U(x, m))
\end{verbatim}
\end{small}


\subsection{Нормальное распределение}

Формула:
\[ Z_1 = \mu \sigma \sqrt{-2 \ln(1 - U_1)} \cos(2 \pi U_2) \]
\[ Z_2 = \mu \sigma \sqrt{-2 \ln(1 - U_1)} \sin(2 \pi U_2) \]

Пример запуска:
\begin{small}
\begin{verbatim}
  python3 main1.py -d nr -f 5p.dat -p1 1024 -p2 1664525 -p3 876  
 \end{verbatim}
\end{small}

Алгоритм:
\begin{small}
\begin{verbatim}
def nr(x, a, b, m):
  y = []
  for i in range(0, len(x)-1, 2):
      base = a + b * np.sqrt(-2 * np.log(1 - U(x[i], m)))
      trig_arg = 2 * np.pi * U(x[i+1], m)
      y.append(base * np.cos(trig_arg), base * np.sin(trig_arg))
  return np.array(y)
\end{verbatim}
\end{small}


\subsection{Гамма распределение}

Алгоритм для $c = k$ ($k$ -- целое число)
\[ Y = a - b \ln \{(1 - U_1) \dots (1 - U_k)\} \]

Пример запуска:
\begin{small}
\begin{verbatim}
  python3 main1.py -d gm -f 5p.dat -p1 1024 -p2 1664525 -p3 876   
 \end{verbatim}
\end{small}

Алгоритм:
\begin{small}
\begin{verbatim}
  def gm(x, a, b, c, m):
  y = []
  u = U(x, m)
  if type(c) == type(1):
      for i in range(0, len(u), c):
          y.append(a - b  * np.log(np.prod(1 - u[i:i+c])))
  else:
      k = int(c - 0.5)
      for i in range(0, len(u), 2 * k + 2):
          z1, z2 = nr([x[i], x[i+1]], 0, 1, m)
          y.append(a + b * (z1 ** 2 / 2 - np.log(np.prod(1 - u[i+2:i+k+2]))))
          y.append(a + b * (z2 ** 2 / 2 - np.log(np.prod(1 - u[i+k+2:i+2*k+2]))))
  return np.array(y)
\end{verbatim}
\end{small}


\subsection{Логнормальное распределение}

Формула:
\[ Y = a + \exp(b - Z) \]

Пример запуска:
\begin{small}
\begin{verbatim}
  python3 main1.py -d ln -f 5p.dat -p1 1024 -p2 1664525 -p3 876   
 \end{verbatim}
\end{small}

Алгоритм:
\begin{small}
\begin{verbatim}
def ln(x, a, b, m):
  return a + np.exp(b - nr(x, 0, 1, m))
\end{verbatim}
\end{small}

\subsection{Логистическое распределение}

Формула:
\[ Y = a + b \ln(\frac{U}{1 - U})\]

Пример запуска:
\begin{small}
\begin{verbatim}
  python3 main1.py -d ls -f 5p.dat -p1 1024 -p2 1664525 -p3 876  
 \end{verbatim}
\end{small}

Алгоритм:
\begin{small}
\begin{verbatim}
def ls(x, a, b, m):
  u = U(x, m)
  return a + b * np.log(u / (1 - u))
\end{verbatim}
\end{small}

\subsection{Биномиальное распределение}
\begin{minted}[fontsize=\footnotesize]{text}
  y = binominal(x, a, b, m):
    u = U(x)
    s = 0
    k = 0
    loopstart:
      s = s + C(k, b) + a^k * (1 - a)^(b - k)
      if s > u:
        y = k
        Завершить
      if k < b - 1:
        k = k + 1
        move to loopstart
      y = b
  \end{minted}
Пример запуска:
\begin{small}
\begin{verbatim}
  python3 main1.py -d bi -f 5p.dat -p1 1024 -p2 1664525 -p3 876  
 \end{verbatim}
\end{small}

Алгоритм:
\begin{small}
\begin{verbatim}
def bi(x, a, b, m):
  y = []
  u = U(x, m)
  for i in u:
      s, k = 0, 0
      while(True):
          s += (factor(b) / (factor(k) * factor(b - k)) 
                          * (a ** k) * ((1 - a) ** (b - k)))
          if s > i:
              y.append(k)
              break
          if k < b - 1:
              k += 1
              continue
          y.append(b)
  return np.array(y)

\end{verbatim}
\end{small}


\newpage
\appendix
    \section{Файлы заданий}
    \subsection{Аргументы для генерации}
    \inputminted[fontsize=\scriptsize]{text}{../tasks/args.py}
    \subsection{Реализация генераторов}
    \inputminted[fontsize=\scriptsize]{text}{../tasks/generators.py}
    \subsection{Реализация распределений}
    \inputminted[fontsize=\scriptsize]{text}{../tasks/distribution.py}
    \subsection{Дополнительные функции}
    \inputminted[fontsize=\scriptsize]{text}{../tasks/utils.py}
    \subsection{main задания 1}
    \inputminted[fontsize=\scriptsize]{text}{../tasks/main.py}
    \subsection{main задания 2}
    \inputminted[fontsize=\scriptsize]{text}{../tasks/main1.py}

\end{document}